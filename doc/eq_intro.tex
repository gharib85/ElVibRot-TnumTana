
\documentclass[a4paper,10pt]{article}
\usepackage{amssymb, amsmath, amsfonts, amsthm}
\usepackage[numbers]{natbib}
\usepackage[utf8]{inputenc}
\usepackage[usenames,dvipsnames]{color}
%\usepackage{tikz}
\usepackage[
      pdftitle={qtransfo},
      pdfauthor={Mamadou Ndong},
      colorlinks=true,
      linkcolor=black,
      urlcolor=black,
      citecolor=black,
      bookmarksopen=false,
      breaklinks=true,
      plainpages=false,
      pdfpagelabels
]{hyperref}
\usepackage[margin = 2cm]{geometry}

% externalization command
\newcommand{\inputTikZ}[1]{%
  \beginpgfgraphicnamed{#1-external}%
    \input{#1.tikz}%
   \endpgfgraphicnamed%
}

\title{Introduction to Tnum and Tana}
%\date{\today}


\begin{document}

%\maketitle
\section{Introduction to the program Tnum-Tana}
The derivation or computation of the kinetic energy operator (KEO) using curvilinear
coordinates for large systems is a difficult task. With this program,
you can calculate the KEO of a given system by using
a numerical approach (Tnum) or an analytical approach (Tana). 
\subsection{Tnum branch}
The Tnum program enables ones to calculate numerically and exactly kinetic
energy operator (KEO), $\hat{T}$, or
the metric tensor (covariant, $\mathbf{g}$, or contravariant, $\mathbf{G}$,
components) using several
kinds of curvilinear coordinates for a molecular system of any size, without
build in limitation.
It can deal with reduced dimensionality KEO (rigid or flexible constraints or
other approximations).

The molecular system has $\textcolor{blue}{N_{at}}$ atoms.
Let $\mathbf{Q}_{dyn} = [Q_{dyn}^1, Q_{dyn}^2, \cdots, Q_{dyn}^{n+m}]$ be the curvilinear coordinates 
associated with the internal deformation.
Where $\textcolor{blue}{n}$ is the number of active coordinates 
and $m$ is the number of inactive coordinates, 
 $n+m = 3N_{at}-6$. 
$n$ can be smaller than $3N_{at}-6$ when
 reduced dimensionality is used, i.e. the quantum dynamics is performed only
 with $n$ coordinates.
 In full dimensionality $m=0$.
The exact KEO, including the deformation, the Coriolis coupling and the overall
rotation is given by,
$$\hat{T} = 
\underbrace{\hat{T}_\mathrm{def}}_\mathrm{Deformation} + 
\underbrace{\hat{T}_\mathrm{cor}}_\mathrm{Corlioris} + 
\underbrace{\hat{T}_\mathrm{rot}}_\mathrm{Rotation} 
$$
 There are several equivalent expressions for the deformation term of the KEO,
 $\hat{T}_\mathrm{def }$, see Eqs.(1-3):
\begin{equation}
  \hat{T}_\mathrm{def}= -\frac{\hbar^2}{2}\sum_{i,j} \Sigma^{ij}(\mathbf{Q}_{dyn})
         \frac{\partial^2 \mathbf{g}}{\partial Q_{dyn}^i \partial Q_{dyn}^j}
         -\frac{\hbar^2}{2}\sum_{i,j} 
         \left(\Sigma^{ij}(\mathbf{Q}_{dyn})
         \frac{\partial ln\left(\rho(\mathbf{Q}_{dyn})\right)}{\partial Q_{dyn}^i}
         +\frac{\partial \Sigma^{ij}(\mathbf{Q}_{dyn})}{\partial Q_{dyn}^i}
         \right) 
         \frac{\partial \mathbf{g}}{\partial Q_{dyn}^j}
         + v_{\mathrm{ext}}(\mathbf{Q}_{dyn})
 \end{equation}
 or

\begin{equation}
  \hat{T}_\mathrm{def}= \sum_{i\le j} f_2^{ij}(\mathbf{Q}_{dyn})
           \frac{\partial^2 \mathbf{g}}{\partial Q_{dyn}^i \partial Q_{dyn}^j}
           +\sum_{j} f_1^{j}(\mathbf{Q}_{dyn})
         \frac{\partial \mathbf{g}}{\partial Q_{dyn}^j}
         + v_{\mathrm{ext}}(\mathbf{Q}_{dyn})
 \end{equation}
 or
\begin{equation}
  \hat{T}_\mathrm{def}= -\frac{\hbar^2}{2}\sum_{i, j} \frac{1} {\rho(\mathbf{Q}_{dyn})}
           \frac{\partial \mathbf{g}}{ \partial Q_{dyn}^i} 
           \left(  \Sigma^{ij}(\mathbf{Q}_{dyn}) \rho(\mathbf{Q}_{dyn})\right)
           \frac{\partial \mathbf{g}}{ \partial Q_{dyn}^j} 
         + v_{\mathrm{ext}}(\mathbf{Q}_{dyn})\,.
 \end{equation}


The Coriolis term, $\hat{T}_\mathrm{cor}$, is given by
$$
  \hat{T}_\mathrm{cor}= -i\hbar \sum_{j, \alpha} \Gamma^{\alpha j} (\mathbf{Q}_{dyn})
  \frac{\partial \mathbf{g}}{\partial {Q}_{dyn}^j} \ddot{\mathcal{P}}_\alpha
  -i\frac{\hbar}{2} \sum_{j, \alpha} \left(\Gamma^{\alpha j} (\mathbf{Q}_{dyn})
  \frac{\partial ln(\rho(\mathbf{Q}_{dyn}))}{\partial {Q}_{dyn}^j}+ 
  \frac{\partial \Gamma(\mathbf{Q}_{dyn})}{\partial {Q}_{dyn}^j} 
  \right)
  \ddot{\mathcal{P}}_\alpha\,.
  $$
The rotational term, $\hat{T}_\mathrm{rot}$, is given by
$$
  \hat{T}_\mathrm{rot}= \frac{1}{2} \sum_{\alpha, \beta} \mu^{\alpha \beta} (\mathbf{Q}_{dyn})
  \ddot{\mathcal{P}}_\alpha
  \ddot{\mathcal{P}}_\beta\,.
$$
Where $i, j = 1, \cdots, n$, $\alpha, \beta = x,y,z$ and $d\tau_\mathrm{def}
= \rho(\mathbf{Q}_{dyn})d{Q}_{dyn}^1 d {Q}_{dyn}^2\cdots{Q}_{dyn}^n$.
\subsection{Tana branch}
The Tana program  is an extension of the Tnum code. The two programs share the same data structure. 
The branch enables to compute the analytical expression of the KEO using the
polyspherical coordinates. 
This expression is expression as,
$$ 
\hat{T} = \frac{1}{2}\sum_{i,j=1}^{N_{at}-1}  M_{ij}
  \vec{\hat{P}}_i^\dagger \cdot \vec{\hat{P}}_{j}\, 
$$
where $\mathbf{M}$ is the mass-matrix
\end{document}
